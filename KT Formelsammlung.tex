\input{header.tex}


\begin{document}

\maketitle

Dieser Text ist unter dieser \href{http://creativecommons.org/licenses/by-nc-sa/4.0/}{Creative Commons} Lizenz veröffentlicht.

\textcolor{red}{Ich erhebe keinen Anspruch auf Vollständigkeit oder Richtigkeit. Falls ihr Fehler findet oder etwas fehlt, dann meldet euch bitte über den Emailkontakt.}

\tableofcontents


\newpage

\section{Beanspruchung}

\begin{align*}
\intertext{Spannung im Balken:}
\sigma &= \frac{F}{A} = Re_p
\intertext{Trägheitsradius:}
i &= \sqrt{\frac{I}{A}} \qquad \text{I = Axials Flächenträgheitsmoment, A = Querschnittfläche}
\intertext{Schlankheitsgrad:}
\lambda &= \frac{l_k}{i} \qquad \text{$l_k$ = Knicklänge}
\intertext{Knicklänge:}
l_k &= k \cdot L_0
\end{align*}

\begin{figure}[h]
	\centering
	\includegraphics[scale=0.7]{Einspannfaelle.jpg}
\end{figure}

\begin{align*}
F_{KE} &= \frac{E \cdot I \cdot \pi^2}{l_k^2} \\ 
\intertext{E = Elas. mod., I = min. axiales Flächenträgheitsmoment}
\intertext{Drucknennspannung bei Knickkraft:}
\sigma_k &= \frac{\pi^2 \cdot E}{\lambda^2}
\intertext{Grenzschlankheitsgrad:}
\lambda_p &= \sqrt{\frac{\pi^2 \cdot E}{Re_p}}
\end{align*}

\subsection*{Schubmittelpunk}


Es gibt folgende Standard Schubmittelpunk Formeln:

\begin{figure}[h]
	\centering
	\includegraphics[scale=0.7]{Schubmittelpunk_1.jpg}
\end{figure}

\begin{figure}[h]
	\centering
	\includegraphics[scale=0.7]{Schubmittelpunk_2.jpg}
\end{figure}

\newpage

Bei jedem anderen Körper rechnet man wie folgt:

\begin{figure}[h]
	\centering
	\includegraphics[scale=0.7]{Schubmittelpunk_Rechnung.jpg}
\end{figure}

\begin{align*}
X_S &= \frac{\sum_{1}^{n} x_n' \cdot A_n'}{\sum_{1}^{n} A_n'} \qquad
Y_S = \frac{\sum_{1}^{n} y_n' \cdot A_n'}{\sum_{1}^{n} A_n'} 
\end{align*}

\subsection*{Querkraft}


\begin{figure}[h]
	\centering
	\includegraphics[scale=0.7]{Querkraftschub.jpg}
\end{figure}

\begin{align*}
\tau_{a,m} &= \frac{F}{b_0 \cdot h} \qquad \qquad \tau_a(z) = \frac{3}{2} \cdot \left[ 1 - 4 \cdot \left( \frac{z}{h} \right)^2 \right] \cdot \frac{F}{b_0 \cdot h}
\end{align*}


\subsection*{Torsion}


\begin{align*}
I_t &= \frac{4 \cdot A_m^2}{\int \frac{\d s}{h(s)}}
\intertext{Für Profile mit abschnittweise konstantem $h(s)$ gilt:}
I_t &= \frac{4 \cdot A_m^2}{\sum_{i}^{} \frac{l_i}{h_i}}
\end{align*}



\subsubsection*{Dünnwandige, geschlossene, einzellige Hohlprofile}


\begin{figure}[h]
	\centering
	\includegraphics[scale=0.6]{Torsion_1.jpg}
\end{figure}

\begin{align*}
\intertext{Der Schubfluss ist über den Umfang konstant:}
t &= \frac{M}{2 \cdot A_m} = const
\intertext{Torsionsspannung:}
\tau_t(s) &= \frac{t}{h(s)} = \frac{M}{2 \cdot A_m \cdot h(s)} 
\intertext{maximale Torsionsspannung:}
\tau_t(s) &= \frac{t}{h(s)} = \frac{M}{2 \cdot A_m \cdot h_{min}} \qquad \text{mit} \qquad W_t = 2 \cdot A_m \cdot h_{min}
\intertext{Verdrillung:}
\phi &= \frac{M \cdot l}{G \cdot I_t}
\end{align*}


\newpage

\subsubsection*{Dünnwandige, geschlossene Profile}


\begin{figure}[h]
	\centering
	\includegraphics[scale=0.6]{Torsion_2.jpg}
\end{figure}


\begin{align*}
I_t &= \frac{4 \cdot A_m^2}{\frac{l_1}{h_1} + \frac{l_2}{h_2} + \frac{l_3}{h_3} + \frac{l_4}{h_4}}
\end{align*}


\subsubsection*{Dünnwandige, geschlossene Profile}


\begin{figure}[h]
	\centering
	\includegraphics[scale=0.6]{Torsion_3.jpg}
\end{figure}


\begin{align*}
\intertext{maximale Torsionsspannung:}
\tau_{t,max} &= \frac{M}{I_t} \cdot h_{max} \qquad \text{mit} \qquad W_t = \frac{I_t}{h_{max}}
\intertext{Verdrillung:}
\phi &= \frac{M \cdot l}{G \cdot I_t} \\
\hfil \\
I_t &= \frac{1}{3} \cdot \sum_{i}^{} l_i \cdot h_i^3
\end{align*}


\subsubsection*{Geschlitzte Rohre}


\begin{figure}[h]
	\centering
	\includegraphics[scale=0.6]{Torsion_4.jpg}
\end{figure}


\begin{align*}
l &= \phi \cdot r \qquad
I_t = \frac{1}{3} \cdot l \cdot h^3 \qquad
W_t = \frac{1}{3} \cdot l \cdot h^2
\end{align*}

\newpage

\subsubsection*{Offene, dünnwandige Profile. Korrekturfaktor.}


\begin{figure}[h]
	\centering
	\includegraphics[scale=0.6]{Torsion_5.jpg}
\end{figure}

\begin{align*}
I_t &= \frac{1}{3} \cdot \eta \cdot \sum_{i}^{} l_i \cdot h_t^3
\end{align*}

\newpage


\section{Mechanismen}

In der Ebene gibt des 3 Freiheitsgrade, im Raum 6. Wegen dem Gestell hat man dann $b \cdot (n-1)$ Freiheitsgrade. Jedes Gelenk eliminiert $u = b - f$ Freiheitsgrade.

\begin{align*}
\intertext{Laufgrad:}
F &= b \cdot \left( n - 1 \right) - g \cdot b + \sum_{i=1}^{g} f_i
F &\leq -1 \qquad \text{überbestimmt, nicht montierbar} \\
F &= 0 \qquad \text{statisch bestimmt} \\
F &= 1 \qquad \text{ein Getriebeglied bewegt auch alle anderen, ein Antrieb} \\
F &\geq 1 \qquad \text{es werden F Antriebe gebraucht}
\end{align*}


\subsection*{Überbestimmtheit}

\begin{align*}
\text{Ü} &= \sum_{i=1}^{k} u_i' - u \\
\text{Ü} &= \text{Grad der Überbestimmtheit} \\
u_i' &= \text{Unfreiheitsgrad des Untergelenks i} \\
u &= \text{Vorgesehender Unfreiheitsgrad des Gesamtgelenks} \\
k &= \text{Anzahl der Untergelenke (Wirkflächenpaare)}
\end{align*}


\newpage

\section{Toleranzen}

\subsection*{Maße}

\begin{figure}[h]
	\centering
	\includegraphics[scale=0.6]{Toleranzen_1.jpg}
\end{figure}

\begin{align*}
M &= N^{A_0}_{A_u} = N + E_c \pm \frac{T}{2} = N^{E_c + \frac{T}{2}}_{E_c - \frac{T}{2}} \\
G &= N + A_o = N + E_c + \frac{T}{2} \\
K &= N + A_o = N + E_c - \frac{T}{2}
\end{align*}

\newpage

\subsection*{Maßtabelle}

\begin{figure}[h]
	\centering
	\includegraphics[scale=0.7]{Masstabelle.jpg}
\end{figure}








\end{document}